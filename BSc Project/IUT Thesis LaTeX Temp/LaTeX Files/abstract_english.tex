\addcontentsline{toc}{section}{چکیده انگلیسی}
\thispagestyle{empty}

\begin{latin}
\begin{center}

{\Huge Increasing Efficiency in Low-Efficiency Systems}

\vspace{1cm}

{\LARGE{Azin Azadeh}}

\vspace{0.2cm}

{\small azin.azadeh@ec.iut.ac.ir}

\vspace{0.5cm}

March 21, 2015

\vspace{0.5cm}

Department of Electrical and Computer Engineering

\vspace{0.2cm}

Isfahan University of Technology, Isfahan 84156-83111, Iran

\vspace{0.2cm}

Degree: M.Sc. \hspace*{3cm} Language: Farsi

\vspace{1cm}

{\small\textbf{Supervisor: Prof. Bahram Borzou (bahram.borzou@cc.iut.ac.ir)}}
\end{center}
~\vfill



\noindent\textbf{Abstract}

\begin{small}
\baselineskip=0.6cm
In most applications, because of numerous advantages it offers,
digital technology (computer, PLC, microcontroller etc.) is used to
control industrial plants. These types of systems, where the process
under control is continuous-time but the controller is digitally
implemented, are called sampled-data systems. Faults can occur in
sampled-data systems like any other control system. In order to
prevent performance degradation, physical damage or failure, faults
should be promptly detected. In this thesis fault diagnosis in
sampled-data systems is studied. The sampled-data design can be
carried out using direct or indirect design approaches. Direct
design, emphasized in this research, does not involve any
approximations.

Normally, to design a robust fault detection and isolation (FDI)
scheme, a performance index which is a measure of the sensitivity of
the FDI to faults and its robustness to unknown inputs and
disturbances is defined and optimized. Different performance indices
based on norms are considered. Using the direct design
approach and the so-called norm invariant transformation, it is
shown that a sampled-data FDI problem can be converted to an
equivalent discrete-time problem. This will form the foundation of a
unifying framework for optimal sampled-data residual generator
design.

Multirate systems are also abundant in industry. Here, several
methods of residual generation based on multirate sampled data are
developed. The key feature of such residual generators is that they
operate at a fast rate for prompt fault detection. The lifting
technique is used to convert the multirate problem into an
equivalent single-rate discrete-time problem with causality
constraints.

It is generally believed that the optimal multirate design performs
better than the optimal slow-rate and worse than the optimal
fast-rate designs. This conjecture is theoretically proved in this
thesis for general multirate control systems with norms of the
closed-loop system as performance indices. However, it is shown that
the common performance indices in FDI design do not satisfy this
property. To resolve this, an alternative performance index is
defined after formulating the FDI problem as a standard control
problem.

\end{small}

\vspace{0.5 cm}

\noindent \textbf{Key Words}: Fault Detection, Wind Turbine Control, Fault Accomodation, Unknown Input Observers

\end{latin}