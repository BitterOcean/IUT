\documentclass{article}
\usepackage[final]{neurips}
\usepackage[utf8]{inputenc} % allow utf-8 input
\usepackage[T1]{fontenc}    % use 8-bit T1 fonts
\usepackage{hyperref}       % hyperlinks
\usepackage{url}            % simple URL typesetting
\usepackage{booktabs}       % professional-quality tables
\usepackage{amsfonts}       % blackboard math symbols
\usepackage{mathtools} 		% mathtools loads the amsmath package automatically
\usepackage{amssymb}
\usepackage{nicefrac}       % compact symbols for 1/2, etc.
\usepackage{microtype}      % microtypography
\usepackage[shortlabels]{enumitem}
\usepackage{xcolor}
\usepackage{graphicx}
\usepackage{fourier}		% numbers sets
\usepackage{xepersian}
\settextfont{XB Yas.ttf}

%------------------------------------------------------------------------------
\title{پروپوزال}

\author{
	مریم سعیدمهر \\
	شماره دانشجویی : ۹۶۲۹۳۷۳
	\and
	ساجده نیک نداف \\
	شماره دانشجویی : ۹۶۳۷۴۵۳
	\and
	مرضیه علیدادی \\
	شماره دانشجویی : ۹۶۳۱۹۸۳
}

\begin{document}	
	%------------------------------------------------------------------------------
	\begin{minipage}{0.1\textwidth}
		\includegraphics[width=1.1cm]{Pic/IUT.png}
	\end{minipage}
	\hfill
	\begin{minipage}{0.9\textwidth}\raggedleft
		دانشگاه صنعتی اصفهان\\
		مهندسی نرم افزار ۲ (نیمسال دوم ۱۳۹۹)\\
	\end{minipage}
	
	\makepertitle
	%------------------------------------------------------------------------------
	\tableofcontents
	\cleardoublepage
	%------------------------------------------------------------------------------
	\section{مقدمه}
	امروزه در اثر فراگیر شدن استفاده از کامپیوتر ها و اینترنت، حجم محتواهای دیجیتال و سرعت تولید آن ها افزایش یافته است و زمان وهزینه ای که برای تولید این محتواهای دیجیتال صرف میشود باعث تبدیل شدن آن ها به کالاهای با ارزشی شده و این تنها در مورد فایل های جدید نیست بلکه در رابطه با فایل های قدیمی که سال هاست تولید کننده ی آن دیگر از آن استفاده نمیکند نیز صدق می کند. \\
	هدف از این پروژه این است که افراد مختلف که صاحبان این فایل های دیجیتال هستند بتوانند به سادگی محتواهای خود را به افراد دیگر عرضه کنند و به خاطر زمان و هزینه ای که صرف تولید این محتوا کرده اند کسب درآمد کنند. همچنین افرادی ک قصد آموزش دیدن دارند یا وقت و تخصص کافی برای تولید محتوای دیجیتال مدنظر خود ندارند به سادگی و با سرچ در میان انبوه از فایل های مشابه بهترین فایل با بهترین کیفیت مدنظر را پیدا کرده و خریداری کنند.
	
	\section{ضرورت پروژه}
	امروزه در ایران و جهان، سایت های مختلفی برای این منظور راه اندازی شده؛ ولی ویژگی مشترک تمام آن ها متمرکز بودن محتوا ها در یک  حوزه ی خاص است. مثلا سایت pinterest.com به خرید و فروش وکتور های تصاویر میپردازد یا سایت ایرانی faradars.org صرفا به خرید و فروش فیلم های آموزشی می پردازد یا بازار های اپلیکیشن صرفا به خرید و فروش اپلیکیشن ها می پردازند. ولی هدف ما طراحی سایتی شبیه به center download است با این تفاوت که کاربران خودشان محتوای سایت را تامین می کنند و بابت آن پول دریافت می کنند.
	
	\section{Workpackage}
	به طور کلی میتوان این پروژه را به شش package work اصلی تقسیم کرد:
	\begin{enumerate}
		\item SignUp
		\item LogIn
		\item خرید
		\item فروش
		\item Seach
		\item شارژ کیف پول
	\end{enumerate}
	
	\section{توجیه اقتصادی}
	هزینه های اولیه ی راه اندازی سیستم، بدین شکل برآورد می شود:
	\begin{itemize}
		\item 500 هزار تومان به منظور راه اندازی سرور
		\item 3 میلیون تومان به منظور تبلیغات اولیه جهت افزایش دامنه ی کاربران
	\end{itemize}
	
	طبق این تخمین اولیه، 3 میلیون و 500 هزار تومان سرمایه ی اولیه جهت راه اندازی سیستم، مورد نیاز است. اگر سود خالص حاصل از تبادلات کاربران را، به طور میانگین 10 هزار تومان در روز در نظر بگیریم، پس از گذشت کم تر از یک سال، سرمایه ی اولیه ی راه اندازی سیستم جبران خواهد شد. و پس از آن، سیستم به سوددهی خواهد رسید.
	
	\section{آینده کاری}
	با توجه به ماهیت این پروژه، انجام آن برای ما بسیار مفید خواهد بود. چرا که پروژه ای کاربردی است؛ و درصورت موفقیت آمیز بودن آن، تجربیاتی از قبیل تعامل با بازار کار، قرار گیری موثر در تیم و انجام کار گروهی، مواجهه با چالش های حوزه ی نرم افزار و ... را به دنبال خواهد داشت.\\
	توجه ! این پروژه قابلیت چاپ و انتشار در قالب مقاله یا گزارش علمی را ندارد.
	
	\section{تبادل دانش}
	با توجه به این که این پروژه را تحت scrum پیش خواهیم برد، در بازه های زمانی کوتاه، سیستم را به صورت یک working software ارائه خواهیم کرد. بدین ترتیب، به صورت پیوسته، از کاربران بازخورد می گیریم و درصدد بهبود عملکرد سیستم و رفع نقص های آن برمی‌آییم. همچنین، با دریافت این بازخورد‌ها می توانیم نیازمندی ها و انتظارات کاربران را بهتر درک کنیم، و اولویت بندی درستی برای پیاده سازی قابلیت های مختلف سیستم در نظر بگیریم.
	
	\section{اعضا}
	اعضای تیم ما به عنوان دانشجویان رشته ی کامپیوتر گرایش نرم افزار، تا کنون در تیم های زیادی تجربه ی همکاری در انجام پروژه را داشته اند. و از لحاظ دانش نرم افزاری نیز در جایگاه خوبی قرار دارند. بنابراین، تیم ما گزینه ی مناسبی برای انجام این پروژه است.\\
	نقش افراد تیم بدین شکل است:
	\begin{itemize}
		\item مدیر پروژه: مریم سعیدمهر
		\item توسعه دهندگان:  
		\begin{itemize}
			\item فرانت اند: مریم سعیدمهر
			\item بک اند: مرضیه علیدادی، ساجده نیک نداف
		\end{itemize}	
		\item ارائه دهندگان: ساجده نیک نداف، مرضیه علیدادی
		\item تحلیل: مرضیه علیدادی، ساجده نیک نداف، مریم سعیدمهر		
		\item تست: مریم سعیدمهر، ساجده نیک نداف
		\item مستند سازی: مرضیه علیدادی
	\end{itemize}
	(همه ی افراد تیم، با روند و نحوه ی انجام تمام فعالیت های مورد نیاز، آشنایی دارند. ولی به طور تخصصی، این نقش ها به افراد اختصاص داده شده است.)
	%------------------------------------------------------------------------------
	\section{تبادل نتایج}
	\begin{enumerate}
		\item حین طراحی و به منظور تبیین دقیق صورت مسئله با استفاده از ارائه prototype گروه‌های هدف را درجریان پیشرفت کار قرار می دهیم.	
		\item پس از ارائه نسخه های مختلف پروژه به صورت فیدبک غیررسمی و یا با استفاده از سیستم پشتیبانی رسمی سامانه (با طراحی و پیاده‌سازی سیستم پشتیبانی فعال تبادلات میان ما و جامعه هدف به صورت پرسش و پاسخ و همچین انتقال انتقادات و فیدبک ها تسریع و تسهیل می شود.)
	\end{enumerate}
	
	%------------------------------------------------------------------------------
	\section{برنامه احتمالی}
	پروژه سامانه خرید و فروش داده دیجیتال قرار است به صورت یک نرم افزار تحت وب باشد. برای توسعه از مدل scrum  استفاده می کنیم.
	\\
در یک ماه ابتدایی کار، پس از جمع ِآوری اطلاعات و انجام مذاکراتی با جامعه ی هدف، backlog product و story user را آماده سازی خواهیم کرد، و جزئیات پیاده سازی را تعیین خواهیم کرد. همچنین، هر سه شنبه نیز جلسه ای بین ساعت ۱۵ الی ۱۷ خواهیم داشت‌.
	\\
پس از تهیه ی این مستندات، sprint ها را آغاز می‌کنیم. sprint های ما به صورت یک هفته ای خواهند بود.
در نهایت، به نظر می رسد که با توجه به‌ اهمیت بخش های مختلف سایت، ابتدا ماژول های logIn و signUp را ایجاد کنیم. پس از آن، به ترتیب، ماژول های خرید و فروش و ماژول سرچ را ایجاد می کنیم. 
	\\
در فاز اولیه، UI ای که تهیه خواهیم کرد، ابتدایی و ساده خواهد بود. و بهبود آن، به فاز های بعدی موکول خواهدشد. و هدف این است که هر بار یک نرم افزاری که کار میکند، ارائه بدهیم.
	\\
پیش بینی اولیه ما از sprint هایی که صرف هر ماژول می شود:

	\begin{itemize}
		\item ماژول های logIn و signUp :  
		 بین ۲ الی ۳ sprint
		
		\item ماژول های خرید و فروش : 
		بین ۴ تا ۵ sprint
		
		\item ماژول search : 
		بین ۴ تا ۵ sprint
		
		\item  ماژول های ارتباط با بانک : 
		بین ۲ الی ۳ sprint
		
		\item بهبود UI :  
		بین ۴ تا ۵ sprint
		
	\end{itemize}
	با توجه به این مسائل، پیش‌بینی ما این است که حدوداً 16 الی 21 sprint نیاز خواهیم داشت.
	در ضمن با این که خروجی هر sprint یک نرم افزار قابل اجرا هست، ولی به نظر میرسد پروژه بعد از sprint دوازدهم قابل عرضه به بازار باشد.
	
	%------------------------------------------------------------------------------
	\section{خطرات}
	از جمله خطراتی که ممکن است سایت را تهدید کند :
	\begin{itemize}
		\item مسائل امنیتی سایت و امکان دسترسی کاربران به فایل های دیگران بدون خریداری  \\
		$ \checkmark $ با در نظر گرفتن مسائل امنیتی و استفاده از فناوری های به روز این مشکل قابل حل است.
		
		\item امکان جذب نشدن مشتری ها به تعداد تخمین زده شده  \\
		$ \checkmark $ احتمالا با کسب دانش و بکار گیری روش های مختلف بازاریابی می‌توان‌ این   مشکل را نیز حل نمود.
		
		\item امکان مواجه شدن با کمبود منابع و حافظه یا پهنای باند  \\
		$ \checkmark $ با ارتقا host این نوع کمبود منابع قابل حل هستند.
	\end{itemize}
	
	%------------------------------------------------------------------------------
	\section{زمان جلسات}
	همه ی اعضای گروه آمادگی برگزاری جلسات در روز سه شنبه بین ساعت 15 تا 17 را دارند.
\end{document}
